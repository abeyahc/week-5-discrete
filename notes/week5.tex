\documentclass{report}

\input{preamble}
\input{macros}
\input{letterfonts}

\title{\Huge{Discrete Mathematics}\\Week 5}
\author{\huge{Abeyah Calpatura}}
\date{}

\begin{document}
\maketitle
\section*{4.1}
\subsection*{Exercises} \\
\text{Abeyah Calpatura} \\
\#7, 13, 24, 28 
\\

\textbf{\#7} 
\sol{There exist real numbers $a$ and $b$ such that $\sqrt{a+b} = \sqrt{a} + \sqrt{b}$.
    \begin{align*}
        & \; a = 16 \; \text{and} \; b = 0  \\
        & \sqrt{16 + 0} = \sqrt{16} + \sqrt{0} \\
        & 4 = 4   
    \end{align*}
}

\textbf{\#13}
\sol{For every integer $n$, if $n$ is odd, then $\frac{n-1}{2}$ is odd.
    \begin{align*}
        & \textbf{Negation:} \; \text{There exists an integer $n$ such that $n$ is odd and $\frac{n-1}{2}$ is even.} \\
        & \textbf{Counterexample:} \; n = 1 \\
        & \frac{1-1}{2} = 0 \; \text{which is even.} \\ 
        & \textbf{Conclusion:} \; \text{The statement is false.}
    \end{align*}
}

\textbf{\#24}
\sol{ For every real number x, if $x > 1$, then $x^2 > x$. 
    \begin{align*}
        & \textbf{Quantification Implicit} \; \text{If x is a real number and $x > 1$, then $x^2 > x$.} \\
        & \textbf{First sentence of proof:} \; \text{"Suppose $x$ is a real number greater than 1."} \\
        & \textbf{Last sentence of proof:} \; \text{"$x^2 > x$"}
    \end{align*}

}

\textbf{\#28}
\sol{
    \begin{align*}
    \textbf{a.} \;
        & \text{$\forall$ integers $m$ and $n$, if $m$ and $n$ are odd, then $m+n$ is even,} \\
        & \text{as $\forall$ odd integers $m$ and $n$, $m+n$ is even,} \\
        & \text{and as If $m$ and $n$ are any odd integers, then $m+n$ is even.} \\
    \textbf{b.} \;
        & \text{By} \; \textbf{definition of odd,} \; \text{$m=2r+1$ and $n=2s+1$ for some} \\
        & \text{integers $r$ and $s$.}\\
        & \text{Then} \\
        & m+n = (2r+1) + (2s+1) \; \text{by} \; \textbf{substitution} \\
        & = 2r+2s+2 \\
        & = 2(r+s+1) \; \text{by algebra}  \\
        & \text{Let $u=r+s+1$. Then $u$ is an integer because $r$, $s$, and 1} \\
        & \text{are integers and because} \; \textbf{definition of}
    \end{align*}
}

\newpage
\section*{4.2}
\subsection*{Exercises} \\
\text{Abeyah Calpatura} \\
\#2, 5, 17, 20 
\\

\textbf{\#2} 
\sol{\text{For every integer $m$, if $m$ is even then $3m+5$ is odd.}
    \begin{align*}
        & \text{By definition of even, $m=2k$ for some integer $k$.} \\
        & 3m+5 = 3(2k)+5 \; \text{by substitution} \\
        & = 6k+5 \; \text{by algebra} \\
        & = 2(3k+2)+1 \; \text{by algebra} \\
        & \text{Let $t=3k+2$. By subsitution,} \\ 
        & 3m+5 = 2t+1
    \end{align*}
}

\textbf{\#5} 
\sol{If $a$ and $b$ are any odd integers, then $a^2+b^2$ is even.
    \begin{align*}
        & \text{Let $a=2r+1$ and $b=2s+1$} \\
        & \text{$a^2+b^2=(2r+1)^2+(2s+1)^2$} \\
        & \text{$= 4r^2+1+4r+4s^2+4s+1$} \\
        & \text{$= 2(r^2+2r^2+2r+2s+1)$} \\
        & \text{Let $t=2r^2+2s^2+2r+2s+1$} \\
        & \text{Then $a^2+b^2=2t$}  
    \end{align*}
}

\textbf{\#17} 
\sol{This proof assumes what is to be proved.} \\

\textbf{\#20} 
\sol{ The product of any two odd integers is odd.
    \begin{align*}
        & \text{$m=2p+1$ and $n=2q+1$} \\
        & \text{$mn=(2p+1)(2q+1)$} \\
        & \text{$=4pq+2p+2q+1$} \\
        & \text{$=2(2pq+p+q)+1$} \\
        & \text{Let $t=2pq+p+q$} \\
        & \text{Then $mn=2t+1$} \\
        & \textbf{True}
    \end{align*}
}

\newpage
\section*{4.3}
\subsection*{Exercises} \\
\text{Abeyah Calpatura} \\
\#5, 15, 21, 36, 38
\\

\textbf{\#5}
\sol{ $0.565656565656...$
    \begin{align*}
        & \text{$x=0.565656565656...$} \\
        & \text{$100x=100(0.56565656...)=56.56565656$} \\ 
        & \text{$100x-x=56.56565656...-0.56565656...$} \\
        & \text{$99x=56$} \\
        & \text{$x=\frac{56}{99}$}
    \end{align*}
}

\textbf{\#15}
\sol{ The product of any two rational numbers ia rational number. 
    \begin{align*}
        & \text{$r=\frac{a}{b}$} \\
        & \text{$s=\frac{c}{d}$} \\
        & \text{$rs=\frac{a}{b}\cdot\frac{c}{d}=\frac{ac}{bd}$} \\
        & \textbf{True} \; \text{$rs=\frac{ac}{bd}$ is a rational number since ac and bd are integers.}
    \end{align*}
}

\textbf{\#21}
\sol{ True or false? If m is any even integer and n is any odd integer, then $m^2+3n$ is odd. Explain.
    \begin{align*}
        & \text{Let $m=2p$ and $n=2q+1$} \\
        & \text{$m^2+3n=(2p)^2+3(2q+1)$} \\
        & \text{$=4p^2+6q+3$} \\
        & \text{$=2(2p^2+3q+1)+1$} \\
        & \text{Let $t=2p^2+3q+1$} \\
        & \text{$m^2+3n=2t+1$} \\
        & \textbf{True}
    \end{align*}
}

\textbf{\#36}
\sol{ 
    \begin{align*}
        & \text{Any two rational numbers have a sum that is rational.} \\
        & \text{Let $r=\frac{a}{b}$ and $s=\frac{c}{d}$} \\
        & \text{$r+s=\frac{a}{b}+\frac{c}{d}=\frac{ad+bc}{bd}$} \\
        & \textbf{True} \; \text{since $ad+bc$ and $bd$ are integers.}
    \end{align*}
}

\textbf{\#38}
\sol{
    \begin{align*}
        & \text{"The sum of two fractions is a fraction" has never been proven.} \\
        & \text{"A rational number is a fraction" is not necessarily true.} \\
        & \text{The statement is false.}
    \end{align*}
}

\newpage
\section*{4.4}
\subsection*{Exercises} \\
\text{Abeyah Calpatura} \\
\#17, 21, 29
\\
\textbf{\#17}
\sol{ For all integers a, b, c, and d, if $a|b$ and $c|d$, then $ac|bd$.
    \begin{align*}
        & \text{By definition of divides, $b=am$ and $d=cn$ for some integers $m$ and $n$.} \\
        & \text{$bd=(am)(cn)=ac(mn)$} \\
        & \text{Let $t=mn$. Then $bd=act$} \\
        & \textbf{True}
    \end{align*}
}
\textbf{\#21}
\sol{ The product of any two even integers is a multiple of 4.
    \begin{align*}
        & \text{Let $m=2p$ and $n=2q$} \\
        & \text{$mn=(2p)(2q)=4pq$} \\
        & \textbf{True} \; \text{since $4pq$ is a multiple of 4.}
    \end{align*}
}
\textbf{\#29}
\sol{ For all integersa and b, if $a|b$ then $a^2|b^2$.
    \begin{align*}
        & \text{By definition of divides, $b=am$ for some integer $m$.} \\
        & \text{$b^2=(am)^2=a^2m^2$} \\
        & \text{Let $t=m^2$. Then $b^2=a^2t$} \\
        & \textbf{True}
    \end{align*}
}

\end{document}
